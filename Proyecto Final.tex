
\documentclass[letterpaper,11pt]{article}
%%%%%%%%%%%%%%%%%%%%%%%%%%%%%%%%%%%%%%%%%%%%%%%%%%%%%%%%%%%%%%%%%%%%%%%%%%%%%%%%%%%%%%%%%%%%%%%%%%%%%%%%%%%%%%%%%%%%%%%%%%%%%%%%%%%%%%%%%%%%%%%%%%%%%%%%%%%%%%%%%%%%%%%%%%%%%%%%%%%%%%%%%%%%%%%%%%%%%%%%%%%%%%%%%%%%%%%%%%%%%%%%%%%%%%%%%%%%%%%%%%%%%%%%%%%%
\usepackage{graphicx}
\usepackage{amsmath,amsfonts,amssymb,amsthm,float}
\usepackage{hyperref}
\usepackage[utf8]{inputenc}
\usepackage[left=2cm, right=2cm, top=2cm, bottom=2cm]{geometry}

\setcounter{MaxMatrixCols}{10}
%TCIDATA{OutputFilter=LATEX.DLL}
%TCIDATA{Version=5.50.0.2953}
%TCIDATA{<META NAME="SaveForMode" CONTENT="1">}
%TCIDATA{BibliographyScheme=BibTeX}
%TCIDATA{LastRevised=Friday, December 13, 2024 18:02:41}
%TCIDATA{<META NAME="GraphicsSave" CONTENT="32">}
%TCIDATA{ComputeDefs=
%1$k_{I}=47202.5937$
%1$k_{P}=67.6355$
%1$k_{D}=0.015336$
%1$C_{r}=1x10^{-6}$
%1$R_{e}=21.\,\allowbreak 185$
%1$R_{r}=1432.\,\allowbreak 9$
%1$C_{e}=1.\,\allowbreak 070\,3\times 10^{-5}$
%}


\input{tcilatex}
\renewcommand{\baselinestretch}{1.15}
\setlength{\parindent}{0pt}
\setlength{\parskip}{0.5\baselineskip}
\pretolerance=2000 \tolerance=3000
\renewcommand{\abstractname}{Resumen}

\begin{document}

\title{Proyecto Final - Sistema Inmunol\'{o}gico}
\author{Ba\~{n}uelos Elias Andres Mart\'{\i}n, Chaparro Zamora Alan Yahir,
Fern\'{a}ndez Esquivel H\'{e}ctor \'{A}ndres \and (21212142, 21212147,
21212153) \\
%EndAName
Departamento de Ingenier\'{\i}a El\'{e}ctrica y Electr\'{o}nica\\
Tecnol\'{o}gico Nacional de M\'{e}xico / Instituto Tecnol\'{o}gico de Tijuana%
}
\maketitle

\noindent \textbf{Palabras clave: }Componentes; Circuito RLC; Estabilidad;
PID; Sintonizaci\'{o}n.

\noindent Correo: \textbf{l21212142@tectijuana.edu.mx;} \textbf{%
l21212147@tectijuana.edu.mx;} \textbf{l21212153@tectijuana.edu.mx}

\noindent \noindent Carrera: \textbf{Ingenier\'{\i}a Biom\'{e}dica}

\noindent Asignatura: \textbf{Modelado de Sistemas Fisiol\'{o}gicos}

\noindent Profesor: \href{https://biomath.xyz/}{\textbf{Dr. Paul Antonio
Valle Trujillo}} (paul.valle@tectijuana.edu.mx)

\section{Funci\'{o}n de transferencia}

\subsection{Ecuaciones principales}

El circuito RLC\ presentado demuestra un comportamiento donde se observa que:

Ve(t) representa la tensi\'{o}n de entrada, que est\'{a} determinada por la
diferencia de corrientes I1(t) y I2 (t), multiplicada por la resistencia R1.
Esto indica que R1 act\'{u}a como un elemento de medici\'{o}n de esa
diferencia de corrientes en el circuito.

\begin{equation*}
V_{e}(t)=R_{1}(I_{1}(t)-I_{2}(t))
\end{equation*}

La derivada de la corriente est\'{a} asociada al comportamiento del
inductor, mientras que la integral de la corriente representa la acumulaci%
\'{o}n de carga en el capacitor.

\begin{equation*}
R_{1}(I_{1}(t)-I_{2}(t))=R_{2}I_{2}(t)+L\frac{dI_{2}(t)}{dt}+\frac{1}{C}\int
I_{2}(t)dt
\end{equation*}

La salida Vs(t) corresponde al voltaje en el capacitor, que est\'{a}
relacionado directamente con la carga acumulada en \'{e}l. En t\'{e}rminos
de la corriente I2(t), este voltaje es proporcional a la integral de la
corriente dividido por la capacitancia C.

\begin{equation*}
V_{s}(t)=\frac{1}{C}\int I_{2}(t)dt
\end{equation*}%
:

\subsection{Transformada de Laplace}

Al realizar la transformada de Laplace en nuestras ecuaciones principales,
obtenemos:

\begin{equation*}
V_{e}(s)=R_{1}(I_{1}(s)-I_{2}(s))
\end{equation*}

\begin{equation*}
R_{1}(I_{1}(s)-I_{2}(s))=R_{2}I_{2}(s)+LsI_{2}(s)+\frac{I_{2}(s)}{Cs}
\end{equation*}

\begin{equation*}
V_{s}(s)=\frac{I_{2}(s)}{Cs}
\end{equation*}

\subsection{Procedimiento algebraico}

Con el objetivo de calcular la funci\'{o}n de transferencia, se realizan
operaciones algebraicas:

\begin{equation*}
V_{e}(s)=R_{1}(I_{1}(s)-I_{2}(s))
\end{equation*}

\bigskip Obteniendo $I_{1}$ de la relaci\'{o}n tenemos:

\begin{equation*}
R_{1}(I_{1}(s)-I_{2}(s))=R_{2}I_{2}(s)+LsI_{2}(s)+\frac{I_{2}(s)}{Cs}
\end{equation*}

\begin{equation*}
R_{1}I_{1}(s)=R_{1}I_{2}(s)+R_{2}I_{2}(s)+LsI_{2}(s)+\frac{I_{2}(s)}{Cs}
\end{equation*}

\begin{equation*}
I_{1}(s)=(R_{1}I_{2}(s)+R_{2}I_{2}(s)+LsI_{2}(s)+\frac{I_{2}(s)}{Cs})\frac{1%
}{R_{1}}
\end{equation*}

Ahora se transcribe en nuestra expresi\'{o}n para tener todo en funci\'{o}n
de $I_{2}$:

\begin{equation*}
V_{e}(s)=R_{1}((R_{1}I_{2}(s)+R_{2}I_{2}(s)+LsI_{2}(s)+\frac{I_{2}(s)}{Cs})%
\frac{1}{R_{1}}-I_{2}(s))
\end{equation*}

\begin{equation*}
V_{e}(s)=R_{1}I_{2}(s)+R_{2}I_{2}(s)+LsI_{2}(s)+\frac{I_{2}(s)}{Cs}%
-R_{1}I_{2}(s)
\end{equation*}

\begin{equation*}
V_{e}(s)=R_{2}I_{2}(s)+LsI_{2}(s)+\frac{I_{2}(s)}{Cs}
\end{equation*}

\begin{equation*}
V_{e}(s)=I_{2}(s)(R_{2}+Ls+\frac{1}{Cs})
\end{equation*}

Eliminando $I_{2}$ de ambos lados de la fracci\'{o}n, se formula que:

\begin{equation*}
\frac{V_{s}(s)}{V_{e}(s)}=\frac{\frac{I_{2}(s)}{Cs}}{I_{2}(s)(R_{2}+Ls+\frac{%
1}{Cs})}
\end{equation*}

\subsection{Resultado}

Habiendo realizado el an\'{a}lisis de la secci\'{o}n, se concluye que la
funci\'{o}n de transferencia del circuito RLC est\'{a} representada por la
siguiente expresi\'{o}n

\begin{equation*}
\frac{V_{s}(s)}{V_{e}(s)}=\frac{1}{CLs^{2}+CR_{2}s+1}\allowbreak
\end{equation*}

\section{Estabilidad del sistema en lazo abierto}

La estabilidad del sistema en lazo abierto se analiza al calcular las raices
del denominador. En el caso de nuestro sistema:

\begin{equation*}
CLs^{2}+CR_{2}s+1=0
\end{equation*}

Los polos del sistema estan dados por lo siguiente:

\begin{eqnarray*}
\lambda _{1} &=&-\frac{1}{2CL}\left( CR_{2}+\sqrt{C^{2}R_{2}^{2}-4CL}\right)
\\
\lambda _{2} &=&-\frac{1}{2CL}\left( CR_{2}-\sqrt{C^{2}R_{2}^{2}-4CL}\right)
\ 
\end{eqnarray*}

\bigskip Con los siguientes valores en los componentes:

\begin{eqnarray*}
R_{2} &=&10 \\
L &=&5\times 10^{-3} \\
C &=&50\times 10^{-6}
\end{eqnarray*}

\begin{eqnarray*}
\lambda _{1} &=&-\frac{1}{2CL}\left( CR_{2}+\sqrt{C^{2}R_{2}^{2}-4CL}\right)
:\text{Complejas conjugadas con negativo }=-1000i\sqrt{3}-1000 \\
\lambda _{2} &=&-\frac{1}{2CL}\left( CR_{2}-\sqrt{C^{2}R_{2}^{2}-4CL}\right)
\ :\text{ Complejas conjugadas con negativo}=1000i\sqrt{3}-1000
\end{eqnarray*}

\section{Modelo de ecuaciones integro-diferenciales}

El modelo de ecuaciones integro-diferenciales esta dado por:

\begin{eqnarray*}
I_{1}(t) &=&(V_{e}(t)+R_{1}I_{2}(t))\frac{1}{R_{1}} \\
I_{2}(t) &=&\int \left[ V_{e}(t)-R_{2}I_{2}(t)-\frac{1}{C}\int i_{2}(t)dt%
\right] \frac{1}{L} \\
V_{s}(t) &=&\frac{1}{C}\int I_{2}(t)dt
\end{eqnarray*}

\section{Error en estado estacionario}

El error en estado estacionario se calcula mediante el siguiente limite:

\begin{equation*}
e\left( t\right) =\lim_{s\rightarrow 0}sR\left( s\right) \left[ 1-\dfrac{%
V_{s}\left( s\right) }{V_{e}\left( s\right) }\right] =\lim_{s\rightarrow 0}s%
\frac{1}{s}\left[ 1-\frac{1}{CLs^{2}+CR_{2}s+1}\right] =0
\end{equation*}

Es decir, el error en estado estacionario es $0V$.

\begin{eqnarray*}
R(s) &:&\ \text{Representa la entrada al sistema.} \\
\dfrac{V_{s}\left( s\right) }{V_{e}\left( s\right) } &:&\text{Representa la
funcion de transferencia del sistema.}
\end{eqnarray*}

\section{C\'{a}lculo de componentes para el controlador PID}

Al sintonizar el controlador PID en Simulink, obtenemos los par\'{a}metros
necesarios para determinar los valores en el circuito f\'{\i}sico

\begin{eqnarray*}
k_{I} &=&\frac{1}{R_{e}C_{r}}=47202.5937 \\
k_{P} &=&\frac{R_{r}}{R_{e}}=67.6355 \\
k_{D} &=&R_{r}C_{e}=0.015336
\end{eqnarray*}

\bigskip Habiendo obtenido los valores de $k_{I}$, $k_{P}$ y $k_{D}$,
proponemos el valor de $1x10^{-6}F$ para el capacitor $C_{r}$. Ahora,
podemos continuar con los calculos de $R_{e}$, $R_{r}$ y $C_{e}$. Obteniendo
as\'{\i}:

\begin{equation*}
C_{r}=1x10^{-6}F
\end{equation*}

\begin{eqnarray*}
R_{e} &=&\frac{1}{k_{I}C_{r}}=21.\,\allowbreak 185 \\
R_{r} &=&k_{P}R_{e}=1432.\,\allowbreak 9 \\
C_{e} &=&\frac{k_{D}}{R_{r}}=1.\,\allowbreak 070\,3\times 10^{-5}
\end{eqnarray*}

\end{document}
